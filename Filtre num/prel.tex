Une fois le signal numérique reçu par l'ADC, il est important de passer ce signal dans un filtre numérique afin de ne conserver que les informations pertinentes (voir Fig. \ref{fig:filtrenumbloc}). Pour ce faire, nous avons dû faire face à différents choix afin de modéliser le filtre de la manière la plus efficace possible.

\begin{figure}[H]
    \centering
    \includegraphics[scale=0.8]{pdffiles/filtrenumbloc.pdf}
    \caption{Schéma-bloc de la partie communication, focalisé sur le filtre numérique. La partie en rouge est détaillée sur la Figure \ref{fig:block_diagram}}
    \label{fig:filtrenumbloc}
\end{figure}

\begin{figure}[h]
    \centering
    \begin{tikzpicture}[node distance=2cm, auto]
        % ADC Block
        \node [block,fill=mygreen] (adc) {ADC};
        
        % Blocks for 900 Hz
        \node [block, right of=adc, node distance=3cm, fill=myblue] (filter1) {Filtrage (900 Hz)};
        \node [block, right of=filter1, node distance=3cm] (window1) {Fenêtre (900 Hz)};
        \node [block, right of=window1, node distance=3cm] (detector1) {Détecteur d'amplitude (900 Hz)};
        
        % Blocks for 1100 Hz
        \node [block, below of=filter1, node distance=2.5cm, fill=myblue] (filter2) {Filtrage (1100 Hz)};
        \node [block, right of=filter2, node distance=3cm] (window2) {Fenêtre (1100 Hz)};
        \node [block, right of=window2, node distance=3cm] (detector2) {Détecteur d'amplitude (1100 Hz)};
        
        % Demodulation Block
        \node [block, right of=detector1, node distance=4cm, text width=8em] (demod) {Démodulation\\ (FskDetector)};

        % Connecting lines for 900 Hz
        \path [line] (adc) -- (filter1);
        \path [line] (filter1) -- (window1);
        \path [line] (window1) -- (detector1);
        \path [line] (detector1) -- (demod);
        
        % Connecting lines for 1100 Hz
        \path [line] (adc) -- (filter2);
        \path [line] (filter2) -- (window2);
        \path [line] (window2) -- (detector2);
        \path [line] (detector2) -- (demod);
    \end{tikzpicture}
    \caption{Schéma-bloc détaillé entre l'ADC et la Démodulation, focalisé sur le filtre numérique}
    \label{fig:block_diagram}
\end{figure}

\subsection{Filtre FIR ou IIR}

\subsubsection{Filtres FIR}

Les filtres FIR sont caractérisés par une structure qui ne fait pas appel à la rétroaction. Leur sortie est calculée uniquement à partir de leur entrée et ils ont l'équation suivante:
\[
y[n] = \sum_{i=0}^{N} b_i x[n-i]
\]
où $b_i$ sont les coefficients du filtre, $x[n]$ est le signal d'entrée, et $N$ est l'ordre du filtre.

Les principaux avantages des filtres FIR sont leur bonne stabilité et la possibilité de réaliser une réponse en fréquence strictement linéaire. Cependant, pour atteindre des spécifications strictes de la bande de transition et de l'atténuation dans la bande d'arrêt, ils requièrent un ordre très élevé, ce qui est peu pratique dans le cadre de ce projet.

\subsubsection{Filtres IIR}

À l'opposé, les filtres IIR utilisent la rétroaction dans leur structure, ce qui les rend plus efficaces qu'un filtre FIR du même ordre pour obtenir une meilleure atténuation. L'équation générale d'un filtre IIR est:
\[
y[n] =  \sum_{j=0}^{N} b_j x[n-j] - \sum_{i=1}^{N} a_i y[n-i]
\]
où $a_i$ et $b_j$ sont les coefficients du filtre, et $N$ est l'ordre du filtre.

Les filtres IIR peuvent être instables si les pôles sont mal placés, mais une bonne conception permet d'éviter ces problèmes.

\subsubsection{Comparaison et Choix}

\textbf{Complexité computationnelle}

Le principal avantage des filtres IIR sur les FIR est leur faible ordre pour une atténuation donnée dans la bande d'arrêt, ce qui se traduit par moins de coefficients et donc moins d'opérations arithmétiques par échantillon traité. Ceci est particulièrement bénéfique dans les applications embarquées où la puissance de calcul et la mémoire sont limitées.

\textbf{Performance en Temps Réel}

La structure récursive des filtres IIR permet une implémentation plus efficace sur des processeurs simples, tels que le dsPIC utilisé pour le robot. Cette efficacité est très importante pour assurer une démodulation rapide et sans perte d'informations.

Le choix entre FIR et IIR dépend aussi des spécificités du système, notamment la tolérance aux phases non-linéaires et la nécessité d'une stabilité absolue. Pour notre robot, la légère phase non-linéaire introduite par un filtre IIR est un compromis acceptable pour bénéficier de sa faible complexité et de sa haute efficacité. Nous avons donc décidé de faire un filtre numérique IIR.

\subsection{Design du filtre numérique}
Pour concevoir les différents étages du filtre numérique, le code Python mis à notre disposition dans le cadre du projet est \textit{digitalFilterDesign.py}. Les filtres passe-bande ont été conçus selon les spécifications suivantes pour chaque fréquence centrale.

\subsubsection{Filtre Passe-Bande à 900 Hz}
\begin{itemize}
\item Fréquence d'échantillonnage : 16000 Hz
\item Fréquence centrale : 900 Hz
\item Largeur de la bande passante : 27 Hz
\item Gain minimum dans la bande passante : 0.9
\item Largeur de la bande bloquante : 63 Hz
\item Gain maximum dans la bande bloquante : 0.1
\end{itemize}

Les coefficients des étages du filtre sont les suivants :
\begin{align*}
    H_1(z) &= 0.001771 \frac{1+2z^{-1}+z^{-2}}{1-1.86370006z^{-1}+0.98824962z^{-2}} \\
    H_2(z) &= 0.001726 \frac{1+2z^{-1}+z^{-2}}{1-1.86719114z^{-1}+0.98840267z^{-2}} \\
    H_3(z) &= 0.023 \frac{1-2z^{-1}+z^{-2}}{1-1.86766577z^{-1}+0.99507105z^{-2}} \\
    H_4(z) &= 0.02283 \frac{1-2z^{-1}+z^{-2}}{1 -1.87591938z^{-1}+ 0.99522511z^{-2}}
\end{align*}
Le gain maximum des étages est 1.6757.

\begin{figure}[H]
    \centering
    \includegraphics[width=0.7\textwidth]{Pictures/filtre iir 900.png}
    \caption{Filtre passe bande et des différents étages pour la fréquence centrale à 900Hz}
    \label{fig:enter-label}
\end{figure}

\subsubsection{Filtre Passe-Bande à 1100 Hz}
\begin{itemize}
    \item Fréquence d'échantillonnage: 16000 Hz
    \item Fréquence centrale: 1100 Hz
    \item Largeur de la bande passante: 33 Hz
    \item Gain minimum dans la bande passante: 0.9
    \item Largeur de la bande bloquante: 77 Hz
    \item Gain maximum dans la bande bloquante: 0.1
\end{itemize}

Les coefficients des étages du filtre sont les suivants:
\begin{align*}
    H_1(z) &= 0.00259 \frac{1+2z^{-1}+z^{-2}}{1-1.80592184z^{-1}+0.98584165z^{-2}} \\
    H_2(z) &= 0.002661 \frac{1+2z^{-1}+z^{-2}}{1-1.80081358z^{-1}+0.98565917z^{-2}} \\
    H_3(z) &= 0.02277 \frac{1-2z^{-1}+z^{-2}}{1-1.8047745z^{-1}+0.99398116z^{-2}} \\
    H_4(z) &= 0.02273 \frac{1-2z^{-1}+z^{-2}}{1-1.8169234z^{-1}+0.99416513z^{-2}}
\end{align*}
Le gain maximum des étages est 1.6796.
\begin{figure}[H]
    \centering
    \includegraphics[width=0.7\textwidth]{Pictures/filtre iir 1100.png}
    \caption{Filtre passe bande et des différents étages pour la fréquence centrale à 1100Hz}
    \label{fig:enter-label}
\end{figure}



\subsection{Implémentation}
\label{res_filtre_num}
Afin de s'assurer de l'implémentation correcte du filtre numérique, celui-ci a été réalisé en 3 étapes :
\begin{enumerate}
\item En Python avec virgule flottante.
\item En Python avec virgule fixe.
\item En C en virgule fixe pour une implémentation directe dans le dsPic.
\end{enumerate}

La première étape nous a servi d'exemple de sorties à avoir dans notre implémentation réelle. Pour ce faire, une sinusoïde a été générée aux fréquences centrales de 900 Hz et 1100 Hz ainsi qu'à 1000 Hz dans le but d'assurer que le signal de la sinusoïde est correctement filtrée.
\begin{figure}[H]
    \centering
    % Première image
    \begin{subfigure}[b]{0.75\textwidth}
        \centering
        \includegraphics[width=\textwidth]{Pictures/graphique_900Hz.jpeg}
        \caption{Sortie de la sinusoide à 900Hz}
        \label{fig:image1}
    \end{subfigure}
    % Deuxième image
    \begin{subfigure}[b]{0.75\textwidth}
        \centering
        \includegraphics[width=\textwidth]{Pictures/graphique_1000Hz.jpeg}
        \caption{Sinusoïde à 1000Hz}
        \label{fig:image2}
    \end{subfigure}   
    % Troisième image
    \begin{subfigure}[b]{0.75\textwidth}
        \centering
        \includegraphics[width=\textwidth]{Pictures/graphique_1100Hz.jpeg}
        \caption{Sinusoïde à 1100Hz}
        \label{fig:image3}
    \end{subfigure}
    \caption{Différentes sorties du filtrage numérique pour des fréquences données}
    \label{fig:subplots}
\end{figure}

Ensuite, les étapes 2 et 3 sont très proches du fait que l'étape 3 est simplement la réimplémentation en C de l'étape 2. Pour ce faire, il a fallu choisir entre précision et temps de calcul. Dans le cadre de ce projet, le processeur étant imposé, le temps de calcul est une contrainte qu'il faut respecter. Différents tests ont été effectués dans le but de connaître nos limites car un grand problème que nous avons rencontré est la gestion d'overflow.

Dans un premier temps, nous avions choisi un décalage de 12 bits (\textit{SCALE\_FACTOR} dans le code), ce qui permettait d'avoir des résultats presque identiques à la simulation en virgule flottante. Cependant, la gestion d'overflow nécessitait une implémentation sur 64 bits pour ne pas perdre de précision. Sachant que la fréquence d'échantillonnage est de 16 kHz, il a fallu s'assurer que le dsPIC puisse effectuer le filtrage dans les deux filtres passe-bandes en moins de $\SI{62,5}{\micro s}$. Cela n'était pas le cas puisque le dsPIC prenait environ $\SI{82}{\micro s}$ sur 64 bits. Le compromis trouvé est d'effectuer les calculs sur 32 bits avec un décalage de 9 bits. C'est également pour réduire le temps de calcul que la fréquence du dsPic a été augmentée à $40MHz$. Les allures des sorties obtenues dans ce cas sont les suivante :

\begin{figure}[H]
    \centering
    % Première image
    \begin{subfigure}[b]{0.7\textwidth}
        \centering
        \includegraphics[width=\textwidth]{Pictures/virgule_fixe_900.png}
        \caption{Sortie de la sinusoide à 900Hz}
        \label{fig:image1}
    \end{subfigure}
    % Deuxième image
    \begin{subfigure}[b]{0.7\textwidth}
        \centering
        \includegraphics[width=\textwidth]{Pictures/virgule_fixe1000.png}
        \caption{Sinusoïde à 1000Hz}
        \label{fig:image2}
    \end{subfigure}   
    % Troisième image
    \begin{subfigure}[b]{0.7\textwidth}
        \centering
        \includegraphics[width=\textwidth]{Pictures/test_virgule_fixe_1100.png}
        \caption{Sinusoïde à 1100Hz}
        \label{fig:image3}
    \end{subfigure}
    \caption{Différentes sorties du filtrage numérique pour des fréquences données}
    \label{fig:subplots}
\end{figure}
Il est possible de voir que le filtre à 900 Hz tend vers une amplitude de $\pm3$ alors que celui à 1100 Hz tend vers $\pm 0.75$, ce qui est tout de même assez différent de notre cas en virgule flottante. Ces différences s'expliquent par la perte de précision du modèle en virgule fixe établi sous les contraintes expliquées ci-dessus. Les valeurs maximales de sortie de ces filtres seront importantes pour le détecteur d'amplitude qui sera expliqué par après.
\subsection{Test Réel}
Une fois le filtre numérique validé. Différents essais ont été réalisé avec le signal reçu par l'ADC comme entrée. Il a fallu s'assurer que le processus prenne moins de $62.5 \mu s$ et que le signal soit également correctement filtré. Pour ce faire, un son aux différentes fréquences a été émis. Les résultats de ces tests sont disponibles sur les figures \ref{fig:filtre1100test}, \ref{fig:filtre900test} et \ref{fig:timefiltre}. Il est important de noter que pour ces résultats, la fenêtre glissante (\ref{fenetre_glissante}) a déjà été implémentée.

\begin{figure}[H]
    \centering
    \begin{subfigure}[b]{\textwidth}
        \centering
        \includegraphics[width=\textwidth]{Pictures/fil900-900.png}
        \caption{Résultat oscilloscope du filtre de 900 Hz pour une entrée de 900 Hz}
        \label{fig:900-900}
    \end{subfigure}
    \begin{subfigure}[b]{\textwidth}
        \centering
        \includegraphics[width=\textwidth]{Pictures/fil900-1100.png}
        \caption{Résultat oscilloscope du filtre de 900 Hz pour une entrée de 1100 Hz}
        \label{fig:900-1100}
    \end{subfigure}
    \caption{Ensemble des résultats pour le filtre 900 Hz}
    \label{fig:filtre900test}
\end{figure}

\begin{figure}[H]
    \centering
\begin{subfigure}[b]{\textwidth}
    \centering
    \includegraphics[width=\textwidth]{Pictures/fil1100-1100.png}
    \caption{Résultat oscilloscope du filtre de 1100 Hz pour une entrée de 1100 Hz}
    \label{fig:1100-1100}
\end{subfigure}

\begin{subfigure}[b]{\textwidth}
    \centering
    \includegraphics[width=\textwidth]{Pictures/fil1100-900.png}
    \caption{Résultat oscilloscope du filtre de 1100 Hz pour une entrée de 900 Hz}
    \label{fig:1100-900}
\end{subfigure}

\caption{Ensemble des résultats pour le filtre 1100 Hz}
    \label{fig:filtre1100test}
\end{figure}

\begin{figure}[H]
    \centering
    \includegraphics[scale=0.35]{Pictures/TimeFIlter.png}
    \caption{Temps de calcul du microcontrôleur pour exécuter les filtres numériques}
    \label{fig:timefiltre}
\end{figure}


\section{Fenêtre Glissante et Seuil d'Amplitude}
\begin{figure}[h]
    \centering
    \begin{tikzpicture}[node distance=2cm, auto]
        % ADC Block
        \node [block,fill=mygreen] (adc) {ADC};
        
        % Blocks for 900 Hz
        \node [block, right of=adc, node distance=3cm, fill=mygreen] (filter1) {Filtrage (900 Hz)};
        \node [block, right of=filter1, node distance=3cm,fill=myblue] (window1) {Fenêtre (900 Hz)};
        \node [block, right of=window1, node distance=3cm,fill=myblue] (detector1) {Détecteur d'amplitude (900 Hz)};
        
        % Blocks for 1100 Hz
        \node [block, below of=filter1, node distance=2.5cm,fill=mygreen] (filter2) {Filtrage (1100 Hz)};
        \node [block, right of=filter2, node distance=3cm,fill=myblue] (window2) {Fenêtre (1100 Hz)};
        \node [block, right of=window2, node distance=3cm,fill=myblue] (detector2) {Détecteur d'amplitude (1100 Hz)};
        
        % Demodulation Block
        \node [block, right of=detector1, node distance=4cm, text width=8em] (demod) {Démodulation\\ (FskDetector)};

        % Connecting lines for 900 Hz
        \path [line] (adc) -- (filter1);
        \path [line] (filter1) -- (window1);
        \path [line] (window1) -- (detector1);
        \path [line] (detector1) -- (demod);
        
        % Connecting lines for 1100 Hz
        \path [line] (adc) -- (filter2);
        \path [line] (filter2) -- (window2);
        \path [line] (window2) -- (detector2);
        \path [line] (detector2) -- (demod);
    \end{tikzpicture}
    \caption{Schéma-bloc détaillé entre l'ADC et la Démodulation, focus sur la fenêtre et la détection d'amplitude}
    \label{fig:block_diagram_fe}
\end{figure}
\subsection{Fenêtre Glissante}
\label{fenetre_glissante}
Pour implémenter la démodulation du signal, nous avons implémenter une fenêtre glissante permettant de choisir la plus grande valeur parmi les échantillons dans la fenêtre. La fréquence d'échantillonnage étant de 16 kHz, il a fallu faire une fenêtre glissante sur $\frac{16000 \text{ Hz}}{900 \text{ Hz}} \approx 17$ échantillons et sur $\frac{16000 \text{ Hz}}{1100 \text{ Hz}} \approx 15$ échantillons. Il est acceptable de considérer une fenêtre glissante identique pour les deux fréquences prenant en compte 16 échantillons. Il est important de considérer que désormais, à la sortie de ce bloc, nous ne travaillons plus à 16 kHz mais bien à 1 kHz, étant donné que la fenêtre glissante ne renvoie que le plus grand échantillon parmi 16. 

\subsubsection{Implementation de la Fenêtre Glissante}
Une fois que les échantillons ont été filtrés, un maximum est déterminé dans une fenêtre glissante de 16 échantillons. Pour chaque échantillon traité, s'il s'agit du premier échantillon de la fenêtre, il est initialisé comme maximum. Pour les échantillons suivants, une comparaison est faite pour déterminer si l'échantillon filtré courant est supérieur au maximum actuel. Si c'est le cas, il devient le nouveau maximum.
\subsection{Seuil d'Amplitude}
Le plus grand échantillon à la sortie de la fenêtre glissante est ensuite comparé à un seuil d'amplitude. Ce seuil est choisi en fonction des valeurs des filtres obtenues lors de la simulation du filtre numérique. Pour rappel, les amplitudes maximales obtenues en virgule flottante sur les graphes pour 900 Hz et 1100 Hz étaient respectivement de 3 et 0.75 (\ref{res_filtre_num}).

Dans le code, la sortie du filtre est en virgule fixe. Par conséquent, des seuils de 500 et de 150 ont été choisis pour les amplitudes à 900 Hz et 1100 Hz respectivement. Ces valeurs ont été déterminées à partir de différents tests visant à ajuster précisément les seuils. L'objectif était de trouver un équilibre entre des seuils suffisamment élevés pour éviter que le bruit ne soit interprété comme un signal, mais pas trop élevés afin de ne pas manquer les véritables signaux. Ainsi, les seuils de 500 pour 900 Hz et de 150 pour 1100 Hz garantissent une détection fiable et précise des signaux. La fonction du seuil d'amplitude renvoie 0 si l'amplitude est inférieur au seuil et 1 pour le cas inverse.





\section{Démodulation}

\begin{figure}[h]
    \centering
    \begin{tikzpicture}[node distance=2cm, auto]
        % ADC Block
        \node [block,fill=mygreen] (adc) {ADC};
        
        % Blocks for 900 Hz
        \node [block, right of=adc, node distance=3cm, fill=mygreen] (filter1) {Filtrage (900 Hz)};
        \node [block, right of=filter1, node distance=3cm,fill=mygreen] (window1) {Fenêtre (900 Hz)};
        \node [block, right of=window1, node distance=3cm,fill=mygreen] (detector1) {Détecteur d'amplitude (900 Hz)};
        
        % Blocks for 1100 Hz
        \node [block, below of=filter1, node distance=2.5cm,fill=mygreen] (filter2) {Filtrage (1100 Hz)};
        \node [block, right of=filter2, node distance=3cm,fill=mygreen] (window2) {Fenêtre (1100 Hz)};
        \node [block, right of=window2, node distance=3cm,fill=mygreen] (detector2) {Détecteur d'amplitude (1100 Hz)};
        
        % Demodulation Block
        \node [block, right of=detector1, node distance=4cm, text width=8em,fill=myblue] (demod) {Démodulation\\ (FskDetector)};

        % Connecting lines for 900 Hz
        \path [line] (adc) -- (filter1);
        \path [line] (filter1) -- (window1);
        \path [line] (window1) -- (detector1);
        \path [line] (detector1) -- (demod);
        
        % Connecting lines for 1100 Hz
        \path [line] (adc) -- (filter2);
        \path [line] (filter2) -- (window2);
        \path [line] (window2) -- (detector2);
        \path [line] (detector2) -- (demod);
    \end{tikzpicture}
    \caption{Schéma-bloc détaillé entre l'ADC et la Démodulation, focus sur la démodulation}
    \label{fig:block_diagram_demo}
\end{figure}

Dans le cadre du projet, la fonction de démodulation nous a été fournie. Les fichiers correspondants sont \texttt{FskDetector.c} et \texttt{FskDetector.h}. Cette fonction est appelée après le bloc du seuil d'amplitude (voir Figure \ref{fig:block_diagram_demo}). Elle prend en paramètres \texttt{detLow} et \texttt{detHigh}, qui correspondent respectivement aux détections des fréquences à 900 Hz et 1100 Hz. Cette fonction détermine tout d'abord l'état du signal comme suit :

\begin{table}[h]
\centering
\begin{tabular}{|c|c|c|}
\hline
\textbf{detHigh} & \textbf{detLow} & \textbf{signalState} \\ \hline
0                & 0               & SILENCE              \\ \hline
0                & 1               & BIT0                 \\ \hline
1                & 0               & BIT1                 \\ \hline
1                & 1               & BRUIT                \\ \hline
\end{tabular}
\caption{États du signal}
\label{tab:signal_states}
\end{table}

Après avoir déterminé l'état du signal, celui-ci est envoyé dans une machine d'état permettant de créer une trame de bits constituée de 10 bits. Ce message créé permet au robot de savoir quelle action il doit réaliser, car chaque trame de bits est associée à un mouvement spécifique.

\subsection{Définition des Paramètres}

Les paramètres de la démodulation sont définis dans le fichier \texttt{FskDetector.h} :

\begin{verbatim}
#define MESSAGE_LENGTH      10
#define SAMPLING_FREQ       1000
#define BIT_FREQ            10
#define OSR                 (SAMPLING_FREQ/BIT_FREQ)
#define FSK_MIN_SAMPLES_NB  (3*OSR/4)
\end{verbatim}

\begin{itemize}
    \item \textbf{MESSAGE\_LENGTH} : Nombre de bits de données dans une trame.
    \item \textbf{SAMPLING\_FREQ} : Fréquence d'échantillonnage du signal audio, en Hz.
    \item \textbf{BIT\_FREQ} : Fréquence des bits de la trame, en Hz.
    \item \textbf{OSR} : OverSampling Ratio, le rapport entre la fréquence d'échantillonnage et la fréquence des bits.
    \item \textbf{FSK\_MIN\_SAMPLES\_NB} : Nombre minimum d'échantillons pour considérer un bit comme valide.
\end{itemize}

% \subsection{Fonction \texttt{fskDetector}}

La fonction \texttt{FskDetector} implémente une machine d'état pour traiter les échantillons et assembler une trame de bits. Voici une explication détaillée du fonctionnement de cette machine d'état :

\begin{itemize}
    \item \textbf{Détermination de l'État du Signal} :
    Le signal est classé en \texttt{SILENCE}, \texttt{BIT0}, \texttt{BIT1}, ou \texttt{BRUIT} en fonction des valeurs de \texttt{detLow} et \texttt{detHigh}.
    
    \item \textbf{Machine d'État} :
    \begin{itemize}
        \item \textbf{IDLE} :
        \begin{itemize}
            \item \textbf{Actions} : Aucune action spécifique.
            \item \textbf{Transitions} : Si un \texttt{BIT0} est détecté, l'état passe à \texttt{START\_BIT} et les compteurs sont initialisés.
        \end{itemize}
        
        \item \textbf{START\_BIT} :
        \begin{itemize}
            \item \textbf{Actions} : Augmenter le compteur de temps (\texttt{timer}) et mettre à jour les compteurs de \texttt{BIT0} et \texttt{BIT1}.
            \item \textbf{Transitions} : Si le compteur de temps atteint \texttt{OSR} et que le nombre de \texttt{BIT0} est suffisant (\texttt{FSK\_MIN\_SAMPLES\_NB}), l'état passe à \texttt{DATA}. Sinon, il retourne à \texttt{IDLE}.
        \end{itemize}
        
        \item \textbf{DATA} :
        \begin{itemize}
            \item \textbf{Actions} : Augmenter le compteur de temps et mettre à jour les compteurs de \texttt{BIT0} et \texttt{BIT1}.
            \item \textbf{Transitions} : Si le compteur de temps atteint \texttt{OSR} et que le nombre de \texttt{BIT0} ou \texttt{BIT1} est suffisant, le bit est ajouté au message et le compteur de bits est diminué. Si tous les bits de données ont été reçus, l'état passe à \texttt{PARITY}.
        \end{itemize}
        
        \item \textbf{PARITY} :
        \begin{itemize}
            \item \textbf{Actions} : Augmenter le compteur de temps et mettre à jour les compteurs de \texttt{BIT0} et \texttt{BIT1}.
            \item \textbf{Transitions} : Si le compteur de temps atteint \texttt{OSR} et que la parité est correcte, l'état passe à \texttt{STOP\_BIT}. Sinon, il retourne à \texttt{IDLE}.
        \end{itemize}
        
        \item \textbf{STOP\_BIT} :
        \begin{itemize}
            \item \textbf{Actions} : Augmenter le compteur de temps et mettre à jour les compteurs de \texttt{BIT0} et \texttt{BIT1}.
            \item \textbf{Transitions} : Si le compteur de temps atteint \texttt{OSR} et que le nombre de \texttt{BIT0} est suffisant, le message est marqué comme complet (\texttt{messageComplete}), et l'état retourne à \texttt{IDLE}.
        \end{itemize}
    \end{itemize}
\end{itemize}

Chaque état effectue des actions spécifiques et vérifie les transitions en fonction des signaux détectés, assurant ainsi la démodulation. La machine d'état permet de traiter les échantillons reçus, de vérifier la validité de chaque bit (données et parité), et de former une trame de bits complète utilisée pour contrôler les actions du robot.
% \begin{figure}[h]
%     \centering
%     \begin{tikzpicture}[shorten >=1pt, node distance=3cm, on grid, auto]
%         \node[state, initial] (idle) {IDLE};
%         \node[state] (start) [right=of idle] {START\_BIT};
%         \node[state] (data) [right=of start] {DATA};
%         \node[state] (parity) [below=of data] {PARITY};
%         \node[state] (stop) [left=of parity] {STOP\_BIT};

%         \path[->]
%         (idle) edge [bend left] node {BIT0} (start)
%         (start) edge [bend left] node {Valid Start Bit} (data)
%         (data) edge [bend left] node {All Data Bits} (parity)
%         (parity) edge [bend left] node {Valid Parity} (stop)
%         (stop) edge [bend left] node {Valid Stop Bit} (idle);
%     \end{tikzpicture}
%     \caption{Diagramme d'état de la fonction \texttt{fskDetector}}
%     \label{fig:fskdetector_state_machine}
% \end{figure}


\section{Conclusion du traitement des signaux audio}

Les sections 3.1 à 3.5 de notre projet montrent l'importance d'un traitement du signal  pour assurer le bon fonctionnement de notre robot. Nous avons dimensionné une chaîne d'acquisition qui remplit le cahier des charges. Ensuite, la conversion analogique-numérique (ADC) a été configurée pour permettre une démodulation rapide sans surcharger le CPU. En choisissant des filtres IIR, nous avons pu obtenir une atténuation efficace avec un ordre minimal, facilitant une implémentation rapide et précise. L'utilisation d'une fenêtre glissante de 16 échantillons pour sélectionner le maximum a permis de réduire la fréquence d'échantillonnage effective, tandis que les seuils d'amplitude ajustés empêchent les faux positifs tout en assurant la détection correcte des signaux. La fonction \texttt{fskDetector} analyse les bits 1 ou 0 des deux fréquences centrales à la sortie du seuil d'amplitude pour déterminer l'état du signal et créer une trame de bits, ce qui permet au robot de savoir quelle action exécuter. Il a fallu que toutes ces étapes soient réalisées en moins de 62.5 $\mu$s pour s'assurer de ne pas perdre des informations du signal.

